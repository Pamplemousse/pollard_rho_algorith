\documentclass{beamer}

\usepackage[french]{babel}    % langue
\usepackage[utf8]{inputenc}   % accents
\usepackage[T1]{fontenc}      % caractères français

\usetheme{metropolis}


\title{L'algorithme $\rho$ de Pollard}
\date{\today}
\author{Laure Bachelet, Charlotte Rance et Xavier Maso}
\institute{Master CSI, Université de Bordeaux}

\begin{document}
  \maketitle

  \begin{frame}{Plan}
    \setbeamertemplate{section in toc}[sections numbered]
	  \tableofcontents[hideallsubsections]
  \end{frame}


  \section{Présentation de la méthode $\rho$}

  \begin{frame}{Problème du Logarithme Discret}
  \end{frame}

  \begin{frame}{Méthode de Pollard}
	  \begin{itemize}
      \item itération (de base, méthode r-adding walks)
      \item collision (lièvre et la tortue, points distingués)
		\end{itemize}
  \end{frame}


  \section{Détection de collisions}

  \begin{frame}{Le lièvre et la tortue}
  \end{frame}

  \begin{frame}{Méthode des points distingués}
  \end{frame}


  \section{Implémentations}

  \begin{frame}{Généralités}
    en C utilisant GMP
  \end{frame}

  \begin{frame}{Tests et mesures}
  \end{frame}

	\begin{frame}{Résultats !}
	\end{frame}

	\section{Conclusion}

	{\setbeamercolor{palette primary}{fg=black, bg=orange!20}
	\begin{frame}[standout]
	  Questions?
	\end{frame}
	}

\end{document}
