\chapter*{Introduction}
\addcontentsline{toc}{chapter}{Introduction}
Le logarithme discret est un objet mathématique de choix pour la cryptographie. En effet, on ne connaît pas de méthode efficace pour le calculer dans le cas général, alors que sa réciproque, l'exponentiation, peut être calculée en un nombre linéaire de multiplications par rapport à l'argument. C'est pourquoi on le retrouve à la base de nombreux cryptosystèmes, tels que El Gamal ou encore l'échange de clefs Diffie-Hellman. Plusieurs algorithmes ont été développés afin de résoudre ce problème - parmi eux, l'algorithme $\rho$ de Pollard, introduit en $1978$, qui fera l'objet de ce rapport.

Dans un premier temps, nous expliquerons le problème du logarithme discret et la version classique de la méthode $\rho$ de Pollard avant d'en présenter notre implémentation. Nous terminerons en étudiant quelques-unes de ses optimisations.
