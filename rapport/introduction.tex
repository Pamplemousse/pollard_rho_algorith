\chapter*{Introduction}
\addcontentsline{toc}{chapter}{Introduction}
Le logarithme discret est une opération mathématique de choix pour la cryptographie. En effet, on ne connaît pas de méthode efficace pour le calculer dans le cas général, alors que sa réciproque, l'exponentiation, peut être déterminée en un nombre linéaire de multiplications par rapport à l'argument. C'est pourquoi on le retrouve à la base de nombreux cryptosystèmes, tels que El Gamal ou encore l'échange de clefs Diffie-Hellman. Plusieurs algorithmes ont été développés afin de résoudre ce problème - parmi eux, l'algorithme $\rho$ de Pollard, introduit en $1978$, qui fera l'objet de ce rapport.

Dans un premier temps, nous expliquerons le problème du logarithme discret et présenterons la version classique de la méthode $\rho$ de Pollard. Ensuite, nous présenterons notre implémentation de cette méthode, codée avec le langage C. Et enfin, nous terminerons en étudiant quelques-unes de ses optimisations. Pour chacune des implémentations que nous avons faites, nous présenterons aussi les performances que nous avons obtenues, et des efforts déployés pour attester de leur bon fonctionnement.
