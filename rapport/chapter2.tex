\chapter{L'algorithme \texorpdfstring{$\rho$}{Rho} de Pollard «\texorpdfstring{\,}{\ }classique\texorpdfstring{\,}{\ }»}
    \section{Présentation}
    % TODO: rédiger une petite intro : présentation de Rho de Pollard dans les grandes lignes.

        \subsection{La fonction d'itération}
        On définit la fonction d'itération comme la fonction permettant de calculer les différents éléments de notre séquence $x_0, x_1, x_2, \ldots$ de la façon suivante~: $x_0 = 1$ et pour $i \geq 0$,

        \begin{align*}
          x_{i+1} =
          \begin{cases}
            \beta \cdot x_i & \text{si } x_i \in S_1 \\
            x_i^2 & \text{si } x_i \in S_2 \\
            \alpha \cdot x_i & \text{si } x_i \in S_3
          \end{cases}
        \end{align*}

        Ainsi, en définissant les suites $(a_i)_{i \geq 0}$ et $(b_i)_{i \geq 0}$ comme suit~: $a_0 = 0$, $b_0 = 0$, et pour $i \geq 0$,

        \begin{align*}
          a_{i+1} =
          \begin{cases}
            a_i                   & \text{si } x_i \in S_1 \\
            2a_i \text{\ mod n}    & \text{si } x_i \in S_2 \\
            a_i + 1 \text{\ mod n} & \text{si } x_i \in S_3
          \end{cases}
        \end{align*}

        \begin{align*}
          b_{i+1} =
          \begin{cases}
            b_i + 1 \text{\ mod n} & \text{si } x_i \in S_1 \\
            2b_i \text{\ mod n}    & \text{si } x_i \in S_2 \\
            b_i                   & \text{si } x_i \in S_3
          \end{cases}
        \end{align*}

        On obtient l'égalité suivante, valable pour tout $i \in \mathbb{N}$ :
        \begin{equation} \label{eq:1}
          x_i = \alpha^{a_i} \cdot \beta^{b_i}
        \end{equation}

        Pour se convaincre de sa véracité, montrons par récurrence qu'elle reste satisfaite pour toute valeur de $i \in \mathbb{N}$.

        \subsubsection{Initialisation}

        Pour $i = 0$, on a par définition~: $x_0 = 1, a_0 = b_0 = 0$

        \begin{align*}
          \alpha^{a_0} \cdot \beta^{b_0} &= \alpha^{0} \cdot \beta^{0} \\
                                         &= 1 \cdot 1 \\
                                         &= x_0
        \end{align*}

        Donc la relation est vérifiée au rang 0.

        \subsubsection{Hérédité}

        On suppose que la relation est vérifiée pour un certain rang $k \in \mathbb{N}$~: $x_k = \alpha^{a_k} \cdot \beta^{b_k}$. Montrons que cela reste vrai au rang $k + 1$, c'est-à-dire que l'on a :

        \begin{align*}
          x_{k+1} = \alpha^{a_{k+1}} \cdot \beta^{b_{k+1}}
        \end{align*}

        On distingue trois cas possibles :

        \underline{$x_{k} \in S_1$}
        \begin{align*}
          a_{k+1} &= a_k \\
          \text{\ et } b_{k+1} &= b_k + 1 \text{\ mod n}
        \end{align*}
        d'où
        \begin{align*}
          \alpha^{a_{k+1}} \cdot \beta^{b_{k+1}} &= \alpha^{a_k} \cdot \beta^{b_k} \cdot \beta \\
                                                 &= x_k \cdot \beta & \text{(par l'hypothèse de récurrence)} \\
                                                 &= x_{k+1} & \text{(car $x_k \in S_1$)}
        \end{align*}


        \underline{$x_{k} \in S_2$}
        \begin{align*}
          a_{k+1} &= 2a_k \text{\ mod n}\\
          \text{\ et } b_{k+1} &= 2b_k \text{\ mod n}
        \end{align*}
        d'où
        \begin{align*}
          \alpha^{a_{k+1}} \cdot \beta^{b_{k+1}} &= \alpha^{2a_k} \cdot \beta^{2b_k} \\
                                                 &= {(\alpha^{a_k} \cdot \beta^{a_k})}^2 \\
                                                 &= {(x_k)}^2 & \text{(par l'hypothèse de récurrence)} \\
                                                 &= x_{k+1} & \text{(car $x_k \in S_2$)}
        \end{align*}

        \underline{$x_{k} \in S_3$}
        \begin{align*}
          a_{k+1} &= a_k + 1 \text{\ mod n} \\
          \text{\ et } b_{k+1} &= b_k
        \end{align*}
        d'où
        \begin{align*}
          \alpha^{a_{k+1}} \cdot \beta^{b_{k+1}} &= \alpha \cdot \alpha^{a_k} \cdot \beta^{b_k} \\
                                                 &= \alpha \cdot x_k & \text{(par l'hypothèse de récurrence)} \\
                                                 &= x_{k+1} & \text{(car $x_k \in S_3$)}
        \end{align*}


        \subsubsection{Conclusion}
        On a montré que pour $i = 0$, l'égalité~\ref{eq:1} est vérifiée.
        De plus, si elle est vraie au rang $k$, alors elle est vraie au rang $k+1$.

        Finalement, pour tout $i \in \mathbb{N}, x_i = \alpha^{a_i} \beta^{b_i}$.
