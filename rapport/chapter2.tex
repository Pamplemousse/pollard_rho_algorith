\chapter{Implémentation de la version classique de l'algorithme \texorpdfstring{$\rho$}{Rho} de Pollard}
    Dans la partie précédente, nous avons discuté de la méthode de Pollard (telle que présentée dans le \textit{Handbook of Applied Cryptography} \autocite[106]{handbook}) pour résoudre le problème du logarithme discret.

    Un des objectifs de ce projet est de produire un programme capable de résoudre le problème, que l'on peut formaliser de la façon suivante.

    En appelant le programme sur des entiers $p, q, g, h$ respectant~:
    \begin{equation} \label{eq:2}
      \begin{split}
        p \text{\ et } q \text{\ deux nombres premiers tels que } {(\mathbb{Z}/p\mathbb{Z})}^* \text{\ soit d'ordre } q \\
        g \text{\ un générateur de ce groupe multiplicatif, \textit{i.e} } {(\mathbb{Z}/p\mathbb{Z})}^* =\ < g > \\
        h \in {(\mathbb{Z}/p\mathbb{Z})}^*
      \end{split}
    \end{equation}

	  Cette dernière condition assure qu'il existe $x \in \mathbb{Z}$ tel que $h = g^x$; ce $x$ correspond à $\log_g(h)$, et c'est ce que l'algorithme retourne.

    Pour des raisons de performances, ce programme est écrit en C.
    Cependant, afin de se familiariser avec la méthode, et nous assister dans la génération de grands nombres $p$, $q$ et d'un générateur $g$ leur correspondant, nous avons commencé par utiliser le logiciel SageMath\footnote{\url{https://www.sagemath.org/}}. De plus, nous avons aussi pu automatiser la génération de données de tests (des entiers $h$ et $x$ respectant $g^x = h$) pour valider le bon fonctionnement du programme, mais aussi mesurer les performances des algorithmes utilisés, afin de pouvoir mettre en perspective d'éventuelles optimisations dont il sera question plus tard.

    Dans ce chapitre, nous allons donc présenter brièvement le premier prototype que nous avons obtenu à l'aide de Sage.
    Bien que non optimisé, cela nous a permis de découvrir quelques questions et contraintes auxquelles nous avons aussi dû répondre lors de l'implémentation en C.
    Nous parlerons de ces problématiques, en particulier de la génération de grands entiers définissant les groupes dans lesquels nous avons travaillé (pour tester et mesurer l'efficacité de notre programme).
    Enfin, nous présenterons le code C ayant permis de générer notre programme de résolution du problème, ainsi que les différents tests et mesures mis en place pour en valider le bon fonctionnement et l'efficacité.


    \section{SageMath}
        \subsection{Prototype de la méthode \texorpdfstring{$\rho$}{Rho}}
        Pour nous aider à prendre en main la méthode, il nous a été proposé de l'implémenter dans un premier temps à l'aide de Sage.

        Nous nous sommes donc appuyés sur les valeurs données dans le Handbook pour tester notre solution.
        Dans les exemples qui vont suivre, nous nous placerons donc dans ${(\mathbb{Z}/383\mathbb{Z})}^*$, d'ordre $191$ et dont $2$ est un générateur.

        Dans les grandes lignes, notre programme Sage s'articule autour de trois fonctions~:
        \begin{itemize}
            \item La fonction \lstinline{f} d'itération.
            \item La fonction \lstinline{rho_table}~: appelle la fonction d'itération jusqu'à détecter une collision (en suivant l'algorithme de Floyd\footnote{Présenté en section~\ref{chapter1:Floyd}.}). Notons que cette fonction stocke l'ensemble des valeurs intermédiaires calculées dans un tableau, pour valider visuellement la correspondance entre la table donnée dans le handbook\autocite[107]{handbook} et les valeurs que l'on obtient.
            \item La fonction de calcul du logarithme discret recherché (sobrement nommée \lstinline{solve}) en fonction des exposants obtenus lors de la collision.
        \end{itemize}

        Notons que ce code est critiquable sur plusieurs points.

        Premièrement, nous n'avons pas cherché à optimiser notre programme, avons mal architecturé les différentes briques logiques, et n'avons porté aucune attention à la gestion mémoire.
        On note donc, parmi les points problématiques de cette implémentation~: mélange des fonctions de calcul et d'affichage et stockage non nécessaire des valeurs intermédiaires (impliquant probablement, dès lors que l'on utilisera de grands nombres, une consommation mémoire démesurée, et des temps de calculs interminables).

        C'est pour contrôler aux mieux ces aspects que nous nous sommes tournés vers le langage C pour implémenter une version efficiente de l'algorithme de Pollard.

        Finalement, obtenir rapidement un prototype de notre solution à l'aide de SageMath nous a permis de soulever ces questions d'organisation du code et d'esquisser l'architecture dont nous aurons besoin lors de l'implémentation en C.
        Mais avant d'entrer dans ces subtils détails, présentons maintenant un autre sujet pour lequel SageMath a été pertinent~: la génération des nombres $p, q, g, h$ respectant les relations définies en \eqref{eq:2}.

        \subsection{Génération de données}
        Pour pouvoir tester et comparer nos futures implémentations en C, nous devons générer un ensemble de données.

        Dans un premier temps, nous allons générer des entiers $p, q, g$ représentant le groupe $\mathbb{Z}/p\mathbb{Z}$ d'ordre $q$ et engendré par $g$, où $p$ et $q$ sont premiers.

        Nous avons choisi de créer un tel groupe grâce à une fonction prenant en paramètres la longueur binaire de $p$ et de $q$.

		    Pour ce faire, on choisit tout d'abord $q$ aléatoirement de $k_q$ bits jusqu'à obtenir un entier $q$ qui soit premier (fonction gen\_order).

		    Ensuite, on construit $p$ tel que $q$ divise $p - 1$, ce qui assurera le fait que $\mathbb{Z}/p\mathbb{Z}$ soit d'ordre $q$. Or on a cela si et seulement s'il existe $u \in\mathbb{Z}$ tel que $p = uq + 1$. Si on veut $p$ de $k_p$ bits, on tire $u$ (toujours aléatoirement) de $k_p - k_q$ bits jusqu'à trouver $p = uq + 1$ premier (fonction gen\_modulus).

		    On obtient ainsi un groupe $\mathbb{Z}/p\mathbb{Z}$ d'ordre $q$.

		    Pour finir, il nous reste à trouver un générateur de ce groupe ainsi formé. On sélectionne un élément $v \in \mathbb{Z}/p\mathbb{Z}$ (concrètement on commence à $v = 2$). On pose alors $g = v^u \text{ mod } p$. Si $g = 1$, on prend un autre élément $v$ (ou on incrémente $v$), sinon $g$ est un générateur du groupe (fonction gen\_group).\\
		    En effet, dans ce dernier cas on aura~:
		    \begin{align*}
		    g^q & = v^{uq} \\
            & = v^{p-1} & \text{ puisque } p = uq + 1 \\
            & = 1 \text{ mod } p & \text{ d'après le petit théorème de Fermat}
		    \end{align*}
		    Et cela garantit que $g$ est un générateur de $\mathbb{Z}/p\mathbb{Z}$ d'ordre $q$.\\

		    Maintenant que nous avons créé un groupe comme nous le souhaitions, nous allons construire, pour un groupe donné, plusieurs entiers (100 dans notre cas) $h$ et $x$ tels que $g^x = h$ mod $p$. Dans la mesure où nous aurons ce $x$, nous pourrons alors vérifier que nos algorithmes de résolution du problème du logarithme discret rendent bien le résultat attendu.

		    Pour cela nous choisissons simplement $x$ aléatoirement, pour ensuite poser $h = g^x$ mod $p$. Notre fonction gen\_data répète ce procédé 100 fois pour ensuite nous retourner ces données.\\

        À présent, grâce à cette dernière fonction, nous pouvons générer un fichier contenant un ensemble de $p, q, g, h, x$ (un par ligne). C'est ce pour quoi nous avons créé la fonction gen\_test\_inputs. Cette fonction construit des groupes d'ordres $q$ de 5 à 55 bits. Pour un $q$ donné de $k_q$ bits, nous prenons un $p$ de $k_q + 10$ bits et avons donc pour chaque couple $(p,q)$ 100 valeurs de $h$ en appliquant la fonction gen\_data.

        Avec ceci, nous pourrons mesurer les performances moyennes de notre programme sur en fonction de la taille de $q$. Dans la dernière section de ce chapitre, nous parlerons finalement de notre implémentation en C de ce programme de résolution du Logarithme Discret, et en présenterons les performances sur le jeu de données que nous venons de générer.
